\documentclass{scrartcl}
\usepackage[misc]{ifsym}
\usepackage[utf8]{inputenc}
\usepackage[T1]{fontenc}
\usepackage{ctex}
\usepackage{xspace}
\usepackage{enumerate}
\usepackage{colonequals}
\usepackage{stmaryrd}
\usepackage{todonotes}
\usepackage{amsmath,amscd,amsbsy,amssymb,latexsym,url,bm,amsthm}
\usepackage{epsfig,graphicx,subfigure}
\usepackage{hyperref}
\usepackage{xcolor}
\usepackage{tikz}
\usepackage{appendix}
\usepackage[vlined,ruled,commentsnumbered,linesnumbered]{algorithm2e}
\usepackage{enumerate,fullpage,proof}
% style adjustments

\renewcommand{\arraystretch}{1.5}

% References

\bibliographystyle{plainurl}
\newcommand{\sref}[1]{{\tiny[#1]}}

% general stuff
\newcommand{\stackl}[2]{\vtop{\hbox{\strut#1}\hbox{\strut#2}}}
\newcommand{\stackc}[2]{\vtop{\setbox0\hbox{\strut #1}\copy0\hbox to\wd0{\hss\strut #2\hss}}}
\newcommand{\stackr}[2]{\vtop{\setbox0\hbox{\strut #1}\copy0\hbox to\wd0{\hss\strut #2}}}

% remember TikZ positions.
\newcommand{\tmark}[1]{
  \tikz[remember picture, baseline, inner xsep=0, inner ysep=0.2em]{ \node [anchor=base] (#1) {\vphantom{M}};
}}%

% listings
\usepackage{listings}
\usepackage{color}
\definecolor{dkgreen}{rgb}{0,0.6,0}
\definecolor{gray}{rgb}{0.5,0.5,0.5}
\definecolor{mauve}{rgb}{0.58,0,0.82}
\lstset{frame=tb,
  language=Python,
  aboveskip=3mm,
  belowskip=3mm,
  showstringspaces=false,
  columns=flexible,
  basicstyle={\small\ttfamily},
  numbers=left,%设置行号位置none不显示行号
  %numberstyle=\tiny\courier, %设置行号大小
  numberstyle=\tiny\color{gray},
  keywordstyle=\color{blue},
  commentstyle=\color{dkgreen},
  stringstyle=\color{mauve},
  breaklines=true,
  breakatwhitespace=true,
  escapeinside=``,%逃逸字符(1左面的键),用于显示中文例如在代码中`中文...`
  tabsize=4,
  extendedchars=false %解决代码跨页时,章节标题,页眉等汉字不显示的问题
}

\newtheorem{thm}{Theorem}

\begin{document}
\title{A search engine for classical Chinese poems}
\author{Xiwei Wu \& Ning Yang \& Hui Ma}

\date{}
\maketitle

\begin{abstract}
  \textbf{Abstract:} This is a group-based project for Web Search \& Mining. In this project, we implemented a poetry crawler that crawled more than 10,000 poems from \href{https://www.shi-ci.com/}{中华诗词网} from pre-Qin to modern times, then implemented a search engine that can support a variety of different search requirements, and finally provided a web front-end that can be easily used. 

\textbf{Keywords: Scrapy, Chinese poems, search engine}
\end{abstract}

\section{Introduction} \label{sec:intro}







\noindent \textbf{Outline.} In Section \ref{sec:crawl}, we will introduce our crawler based on Scrapy and SQL. In Section \ref{sec:search}, we will introduce our design for search engine. In Section \ref{sec:web}, we will introduce our design for web front-end. In Section \ref{sec:concl}, we will give a demonstration of the results of our project.
\section{Data Crawling} \label{sec:crawl}

We use Scrapy for crawling data and SQL for storing data. Here we use \href{https://www.shi-ci.com/}{中华诗词网} as our poem source.
Our crawler starts with the pages belonging to different dynasties. By analyzing the web code we get the relevant information.
\begin{lstlisting}
  def start_requests(self):
      dynastys = {
          "先秦": "https://www.shi-ci.com/dynasty/72057594037927936", 
          "汉代": "https://www.shi-ci.com/dynasty/144115188075855872", 
          "三国两晋": "https://www.shi-ci.com/dynasty/216172782113783808", 
          "南北朝": "https://www.shi-ci.com/dynasty/288230376151711744", 
          "隋代": "https://www.shi-ci.com/dynasty/360287970189639680", 
          "唐代": "https://www.shi-ci.com/dynasty/432345564227567616",
          "宋代": "https://www.shi-ci.com/dynasty/504403158265495552", 
          "元代": "https://www.shi-ci.com/dynasty/576460752303423488", 
          "明代": "https://www.shi-ci.com/dynasty/648518346341351424",
          "清代": "https://www.shi-ci.com/dynasty/720575940379279360", 
          "近现代": "https://www.shi-ci.com/dynasty/792633534417207296",
        }
      for k,url in dynastys.items():
          yield Request(url,meta={"item":k},callback=self.parse_dynasty)
\end{lstlisting}

For each poem, we collect dynasty, poet name, poem name, contents, poet description and the associated page url.
\begin{lstlisting}
  class PoemItem(scrapy.Item):
    dynasty = scrapy.Field()
    poet_name = scrapy.Field()
    poet_url = scrapy.Field()
    poem_name = scrapy.Field()
    poem_url = scrapy.Field()
    contents = scrapy.Field()
    poet_desc = scrapy.Field()
    crawl_url = scrapy.Field()
\end{lstlisting}

Based on these data, we created the table via mysql, and the relevant steps can be referred to the poem/READMD.md file. Finally, we output the data in SQL format for later use. You can see our dataset as poem/data/poem.sql.

Due to the problem of website data, we need to process the poems obtained by the crawlers, such as the unification of Chinese and English commas, the addition of missing content in the body, and the appearance of ellipses due to the excessive length of titles. Here we would like to mention the ellipsis in the title, we found that among the more than 50,000 ancient poems we collected, there are more than 1,200 poems with ellipsis in the title, most of them are Tang poems. And we found that the titles of these poems existed in most of the websites in the form of containing ellipses, so we also dealt with only some of the poems for which the full names could be found.
\section{Search Engine} \label{sec:search}

\subsection{Overall Design}

\subsection{Boolean Search}

\subsection{Zone-specific Search}

\subsection{Pinyin-based Tolerant Search}

\subsection{Ranked Search}


\section{Web Interface} \label{sec:web}


\section{Conclusion} \label{sec:concl}

You can see our codes and dataset on the Github \href{https://github.com/yashen32768/Crawl-for-poems}{Crawl-for-poems}.

\end{document}